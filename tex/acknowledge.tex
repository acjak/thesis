\chapter{Acknowledgements}
First of all I would like to thank my wife and family for their patience and support during my PhD work. I promise to return to the living now.

A lot of people have been extremely helpful during my PhD work. First of all I would like to thank my office mate and colleague Desiree D. M. Ferreira for great project management, emotional support and friendship during the past four years.

Many problems in the lab could not have been solved without the help from Joan Momberg, Michael H. Avngaard and Kim E. Madsen from the electronics department at DTU Space. Joan has been an expert repository of knowledge and experience in the coating facility.

I am very thankful for the support from Birte E. Hede and Lene Bettenhaus from DTU Space for whenever I needed administrative help.

A great thanks goes to Michael J. Pivovaroff from Lawrence Livermore Natl. Lab. for giving me the opportunity to have my external stay there and inviting our group into the CAST collaboration. The CAST XRT would not have happened without him and the people from his group, in particular Todd Decker, Jaime Ruz and Julia Vogel. Also thanks to Regina Soufli and Chris Walton for nice discussions.

I would also like to thank Igor Irastorza, Juanan Garcia, Javier G. Garza and Juan Castel from University de Saragoza for their great support in making the CAST telescope installation happen and designing detectors and X-ray source.

Also a big thanks to Igor and Mike for coming to Denmark and talking at the small axion symposium I arranged.

A good number of measurements were done at the BESSY II synchrotron at PTB in Berlin. I would like to thank Stefanie Marggraf, Levent Cibik and Michael Krumrey for the work and patience they put into the measurement campaigns.

New people in the X-ray optics group at DTU Space will be taking over my responsibilities in the coating facility and I would like to thank Sonny Masahi and David Girou for doing this and at the same time wish them good luck. David Girou also performed the ISO qualification tests for the Athena coatings, which has been a huge help.

Lastly, I would like to thank my supervisor Finn E. Christensen for letting me pursue all the projects I wanted. The PhD work has been exciting and challenging, primarily as a result of Finn's long experience in the field which constantly opened up to new opportunities.
