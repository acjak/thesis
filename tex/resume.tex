\chapter{Resum\'e} %{Sammenfatning (Summary in Danish)}
\selectlanguage{danish}
Til at opdage nye fænomener i fysikken kræves evigt større og mere avancerede instrumenter for at detektere enten fundamentale partikler eller energirige begivenheder i universet. Denne afhandling beskriver arbejdet på tre separate Röntgenteleskop, én til astrofysik og to til astropartikelfysik. Alle gør de brug af lavvinkelreflektioner på Röntgenspejle med specielle belægninger.

Reflekterende belægninger der gør brug af forskellige materialekombinationer på Silicon Pore Optic (SPO) substrater blev undersøgt til den store europæiske Athena Röntgenteleskopsmission. Belægninger med Ir/B$_4$C enkeltlag eller multilag blev karakteriseret og kvalificeret for stabilitet over længere tid samt reflektionseffektivitet. Et Cr underlag under Ir/B$_4$C belægninger viste sig at begrænse tyndfilmsstress og derudover viste Ir belægningen at have en glattende effekt når deponeret ovenpå en ru Cr overflade. Opskalering af produktionen af reflekterende belægninger på 210,000 SPO substrater over en to års periode er diskuteret og en multikammersløsning er beskrevet.

Et Röntgenteleskop til CAST helioskopet på CERN blev designet, optimiseret, produceret og installeret for at forbedre følsomheden af helioskopet. Det installerede teleskop fokuserer Röntgenstråler, konverteret fra axioner ved Primakoff-effekten, i et detektorareal 400 gange mindre end før. Målinger med en Röntgenkilde viste at teleskopet opfører sig som udregnet med raytracing.

En efterfølger til CAST helioskopet kaldet International AXion Observatory (IAXO) er i defineringsfasen og Röntgenteleskoper til det meget større instrument blev designet. Software blev udviklet til at udregne den optimale fokallænge baseret på den estimerede figurfejl i teleskopet og den angulære størrelse på solen.

En ny softwareløsning til coatingfaciliterne på DTU Space blev udviklet ved brug af SPEC software pakken for at forbedre fleksibiliteten og mulighederne af opstillingen. Instrumenterne forbundet coatingkammeret blev alle implementeret. Multilagsbelægninger til CAST Röntgenteleskopet samt Athena kvalificeringerne blev foretaget ved brug af det nye software.

\selectlanguage{english}
