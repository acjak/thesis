\chapter{Summary}
Discovering new phenomena in physics require ever larger and more advanced instruments in order to detect either fundamental particles or energetic events in the universe. This thesis describes the work done on three separate X-ray telescopes, one for astrophysics and two for astroparticle physics; all of which makes use of grazing incidence reflecting X-ray optics.

Reflective coatings using various materials on Silicon Pore Optic (SPO) substrates were investigated for the European Athena large X-ray telescope mission. Ir/B$_4$C single bilayer and multilayer coatings were characterised and qualified for long term stability and reflectivity performance. A Cr sublayer under an Ir/B$_4$C coating was seen to mitigate the film stress; additionally, Ir coatings were found to show a smoothening effect when deposited onto a rough Cr surface. The coating production upscaling to 120,000 SPO substrates coated over a two year period is discussed and a multi chamber solution is described.

An X-ray telescope for the CAST helioscope at CERN was designed, optimised, produced and installed in order to improve the sensitivity of the helioscope. The installed telescope focuses X-rays, converted from axions through the Primakoff effect, into a detector area 400 times smaller than before. Measurements using an X-ray source shows the telescope behaving as calculated through ray tracing.

A successor to the CAST helioscope named the International AXion Observatory (IAXO) is in the definition phase and X-ray telescopes to the much larger instrument was designed. Software was developed to calculate the optimal focal length based on estimated telescope figure error and angular size of the sun.

A new software solution for the coating facility at DTU Space was developed using the SPEC software package, improving the flexibility and capability of the setup. The instruments connected to the coating chamber were all implemented. Coatings for the CAST X-ray telescope and Athena coating qualifications were done using the new software.
