\chapter{Introduction}
For the past 50 years, the quest for discovering the cosmos has led to a large array of advances in vision and optics based technology. To overcome the challenges of imaging objects far away and at wavelengths way beyond the capabilities of the human eye, scientists and engineers have come up with ingenious solutions. Specifically, when it comes to imaging and detection in the X-ray energy range (\~0.1-500 keV), whole new obstacles had to be overcome.

The first problem was the inability of X-rays to penetrate Earth's atmosphere, which necessitated detectors to be placed on baloons, sounding rockets or in orbit around earth. That gave rise to a drive for weight saving, miniturization and power conservation.

The next problem that became apparent was the inability of X-rays to be reflected in a mirror like in ground based optical telescopes or orbiting infrared telescopes such as Hubble. The high energy of X-rays means that the refractory index of a material become less than unity, so instead of being reflected, the photons are absorbed. However, there is a workaround: The X-ray photons can be reflected at very low grazing angles, e.g. a 10 keV photon can be reflected at up to \~0.5 degrees from a gold surface with almost 100\% intensity. Luckily, the gold is not strickly necessary as any high electron density material will do (anything with a high Z number in the periodic table). Then how can we make an optic that reflects at such low grazing angles, but still has a big collecting area, and preferably also focuses like a parabolic mirror? The answer came from nested shells of concentric mirrors all angled\footnote{It is important to consider that in astrophysical observersations, the photons coming from a distant object are described as completely collimated or in other words: Their trajectories are parralel so all arrive at the optic with the exact same normal angle.} to reflect incoming X-rays to the same spot. Using two sets of mirrors, the first with a parabolic shape and the second with a hyperbolic shape will make it possible to fit the optic in a spacecraft that will fit on a rocket. The design is called a Wolter I type optic (named after the inventor by the inventor) and specifically addresses the needs for a grazing angle focusing telescope.

Then a new problem comes to light: As the energy of the X-ray photon increases, the grazing angle at which we see reflection from a gold surface decreases dramatically. That leaves us with two options, either increase the length of the optic with extendable masts (costly and technically difficult), or somehow improve the reflecting surface. Looking at the properties of X-rays it was seen that it is possible for an X-ray photon to reflect from the lattice plane of a crystal. Specifically, the photon saw the change in electron density from between the lattice planes to a lattice plane like a surface.\footnote{The reverse argument is more correct. Photons reacting to a surface are just photons reacting to a change in electron density.} Additionally, the photons achieve constructive interference at certain angles related to the photon wavelength and lattice spacing known as the Bragg condition\footnote{$n\lambda=2d\sin(\theta)$ with $\lambda$ being the wavelength, $d$ the lattice spacing and $\theta$ the angle}. So how to take advantage of those X-ray properties? Using lattices means using perfect crystals and shaping them into a Wolter I type optic, both of which creates all new problems\footnote{Using perfect crystals for X-ray and Gamma-ray instrumentation are being investigated and are called Laue lenses.}. Instead the attention was turned to thin film coatings. Advances in technology made it possible to deposit extremely thin and very uniform coatings with a wide variety of materials. By applying a multilayer coating of interchanging materials with low electron density and high electron density, a pseudo crystal can be created. The thickness of each layer can be determined precisely, so in accordance with the Bragg condition the thickness can be designed to reflect at a given angle and photon energy. The constructive interference of the Bragg condition is however a drawback when it comes to astrophysical observations, as only a small bandwidth of photons will be reflected. To overcome that problem, a multilayer with hundreds of layers and film thicknesses that varied from top to bottom was developed. These coatings are called graded-d multilayers and was used for the first time in an astrophysical observatory in NuSTAR, which was launched in 2012. NuSTAR has from 2012 to 2014 been the NASA mission with the second most published papers from the observations.

In this thesis is described work done from 2011 to 2014 on coating developments for ATHENA\footnote{The ATHENA mission is a large ESA X-ray observatory selected for the L2 mission in 2028. It has quite a long history of cross-agency collaboration back when it was called IXO, but that was unfortunately ended by the effects of the financial crisis.} and design, production and installation of the CAST\footnote{CAST is a CERN collaboration to look for the theoretical axion particle.} X-ray optic. Both those instruments function in an energy band in the very low end compared to NuSTAR, right in the region where there is only a limited benefit from using multilayers compared to single or double layer coatings.

For the coating developments for ATHENA (chapter \ref{chap:athena_coatings}), the main goal was to find coatings that are stable and well-performing. Coatings that will behave well even after being launched by a rocket and drifting around a Lagrange point for 10+ years. At the same time the coatings should be able to reflect well enough to give ATHENA the largest effective area of any X-ray telescope. A new optics technology was developed by ESA leveraged by advances in the semiconductor industry, but require specific processes that would ruin pure carbon coatings, an element used in the NuSTAR coatings. The final problem was finding a way to coat the 300,000 mirrors required for the mission in the 2 years allocated by ESA for mirror production and coating.

\section{X-ray optics described}
\section{X-ray telescopes described}
\subsection{This is a new subsection}
