\chapter{Changelog}

\begin{itemize}
  \item Switched the two figures in fig 3.14 to correspond with the later figures. Also corrected the text.
  \item Added text to figures in fig 3.16 using overpic.
  \item Chapter on X-ray principles inserted. There might be adjustments needed. Diffuse reflection section could be taken out (Now out 14/3). Maybe a reference to master thesis?
  \item Section on multilab software moved to appendix. Might need adjustments. There should be a reference in the multilab chapter.
  \item Included side view illustrations of CAST coating recipes along with a table.
  \item Increased size of two first figures in CAST chapter.
  \item Changed amount of SPOs in summary and resume seactions.
  \item Changed caption formatting to differentiate from regular text.
  \item Increased size of fig. 3.15 (ruined sample). Added more description to the caption.
  \item Plot of pulsed-DC sputtered samples measured with XRR redone to be more clear.
  \item Long term stab. plots increased font size and added description on plots.
  \item Removed the first half page of the paper in app. D. Original paper published under Creative Commons license and can therefore be reprinted in the thesis.
  \item Added more text about the Primakoff effect.
  \item Increased image sizes to 8 cm in sections 3-3.1
  \item Moved subsections on Ir/B4C further up in Athena telescope section.
  \item Changed headline of section 3.2 to reflect it being about film stress.
  \item Added to paragraph of stress the results that stress can have in films.
  \item New subsection for Cr roughness investigation.
  \item Added diagram of NuSTAR glass dimensions
  \item Added illustration of NuSTAR optic.
  \item Changed roughness symbol in introduction to $\sigma$ from $\theta$.
  \item Added astrophysical motivation for Athena mission in the beginning of the Athena chapter.
  \item Added some details on ISO tests.
  \item Added subsection describing effective area to introduction.
\end{itemize}

\section{Pending changes}
\begin{itemize}
\item Figure texts lacking.
\end{itemize}
\subsection{Athena stuff}
\begin{itemize}

	\item \st{ Athena production section. Show real values.}
	\item \st{Athena baseline explanation earlier in the chapter. Why Ir/B4C?}
	\item Stress study. Hard to see connection. Why?\\
      -Sæt beskrivelse af stress påvirkning af data ind i introduktionen til sektionen for at lægge op til hvad kapitlet vil handle om.
	\item \st{Why do we reflect in X-ray? (What's interesting to see?)}\\
      -Baseret på Athena proposal beskriv grunde til hvorfor Athena skal kigge i blød del af x-ray området. Hvad forventer man at se og hvorfor er det nyt og interessant. Fe emittere ved 6.5 keV. Man kan se alt lettere end jern. F.eks. kan man se progenitor stjerner for supernova rester (dvs. man kan se hvilken type stjerne der er sprunget i luften og har dannet supernova resten, da gassen der stammer fra stjernen er meget varm og udsender blød x-ray stråling.
	\item \st{Section on Athena qualifications, too much inside knowledge, why ISO for instance?}\\
         - Bulletpoints with reasons?
	\item Explanation on cable length\\
      -It has been mentioned at several SPIE conferences that the cable length could be a problem and could induce cathodic arcing and influence the coating quality. No references to written documentation was provided at the time, and this should therefore not be...
	\item Conclusions lacking on pulsed-DC etc.
	\item \st{Picture of broken sample unclear.}
	\item Need figure of measurement positions at BESSY
	\item Athena prod. Why witness samples?\\
      -To performe quality control during the production phase in order to ensure that the quality is at its maximum potential as determined in the design and optimization phases, and to avoid potential aggregation of systematical errors.

\end{itemize}

\subsection{Introduction stuff}
\begin{itemize}
  \item \st{Effective area. Show somewhere.}\\
      -In order to get as sensitive a telescope as possible, and in order to catch the relatively sparse amount of soft X-ray photons from far away sources, an effective area as large as possible is needed. For this reason it is important to design an optimize the telescope optics so that this is accomplished. Since the design and geometry of the optics module was already decided, it is lastly the reflective surface of the individual mirrors which should be optimised in order to achieve the best effective area. This means that both the material combinations, the thickness of the coating, the number of bilayers, and the use of different recipes throughout the optics design will have an influence of the telescope throughput. In this thesis I will investigate different combinations of the above and present the results with the purpose of identifying the best possible coating that will help us achieve the described objectives.
  \item First figure in introduction. Too much stuff.\\
      -Ikke grafer og plots i introduktionen. Introduktionen outliner problemet og beskriver hvordan problemet vil blive forsøgt løst. (populærvidenskabeligt med lidt krymmel på toppen)
  \item \st{NuSTAR mirror figure.}\\
      -Needs fig. text and reference.

%  \begin{figure}[htbp]
%    \begin{asy}
  %    size(8cm,0);
%      draw(arc((-0.5,0), r=0.8, angle1=-120, angle2=-60));
%      draw(arc((-0.5,-0.01), r=0.8, angle1=-120, angle2=-60));
%    \end{asy}
%  \end{figure}


\end{itemize}


\subsection{CAST stuff}
\begin{itemize}
  \item \st{ Show coating parameters.}
      - Need pressure parameters?
  \item \st{Primakoff effect need explanation.}
  \item Section on Micromegas.\\
      -In this thesis the focus is on the optics part of both the Athena and the Cast project. The theory, issues, and scientific background for the design of the detector and other features of the two projects will not be treated. I have in this work made use of a few select constants related to the two detectors, that being the quantum efficiency used in the optimisation of the coatings and the focal length, i.e. the length between the optics module and the detector module.
\end{itemize}

\subsection{IAXO stuff}
\begin{itemize}
  \item \st{ Remove journal paper and write into the chapter instead?}
  \item Make sure the chapter makes sense in connection with the paper in appendix D.
\end{itemize}
