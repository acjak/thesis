%!TEX root = /Users/andcj/Dropbox/Documents/PhD/Thesis/phdthesis.tex
\chapter{Conclusion and outlook}
In this thesis I have described investigations into grazing angle type X-ray optics for astrophysics and astroparticle physics, in particular I have during the PhD study worked with the following:

\begin{enumerate}
  \item Developed new software to control the coating chamber at DTU Space.
  \item Characterised and qualified coatings for the European Athena mission.
  \item Investigated production facility requirements to coat flight substrates for Athena.
  \item Designed, produced and installed an X-ray telescope for the CAST helioscope at CERN.
  \item Designed X-ray telescopes for the proposed International AXion Observatory (IAXO).
\end{enumerate}

In the first chapter of the thesis I have described the coating facility at DTU Space and the former software solution used to control the multilayer coating deposition process for the past decade. A number of features were missing in the old software, such as continuous logging and customisation options. A new software solution was produced using the SPEC package to address the missing features. At the time of writing, the software has been controlling the chamber for $\sim$1 year and been capable of producing complicated NuSTAR-like coatings as well as several coatings in a single run while varying e.g. power supply pulse frequency or chamber pressure. A continuous log of chamber pressure and power supply output is updated every 5 seconds and kept on the computer. A webpage has been set up to show plots of pressure and power output for the preceding three days and is accessible from any internet-connected computer.

Investigations into baseline and optimised coatings for the Athena mission led to a characterisation of the B$_4$C-based multilayer material combinations Pt/B$_4$C, Ir/B$_4$C and W/B$_4$C. Long term storage investigations showed instabilities in Pt/B$_4$C and W/B$_4$C at ambient conditions, where interdiffusion and reaction to humidity changed the multilayer structure over time. Ir/B$_4$C baseline and optimised multilayer coatings showed complete stability in the qualification tests.

Deposition of coatings using pulsed-DC sputtering and reactive sputtering were investigated for the Athena mission. The two methods showed no improvements in interlayer roughness. Coatings of a single iridium layer with chromium sublayer has shown a significant improvement in iridium surface roughness compared to the Ir/Cr interface roughness.

To scale production of coated SPO substrates to the 210,000 needed for Athena within a 2 year timespan, possible manufacturing methods were investigated. A solution with 2-3 multi chamber sputtering instruments were found to be the optimal solution to meet the requirements.

An X-ray telescope was designed for the CAST helioscope using custom software to find optimal geometry and reflective multilayer coatings, taking into account the detector quantum efficiency and axion spectrum. Surplus NuSTAR glass substrates were coated with optimised multilayer coatings, which were subsequently assembled into a Wolter I type telescope at Lawrence Livermore National Lab. The telescope was installed on the CAST helioscope at CERN and aligned using a theodolite and a gun-sight construction on the optic. Lastly, a pyroelectric X-ray source was used to check the alignment of telescope and detector; the resulting detector output matches a ray-tracing calculations of the setup.

The software used to design and optimise the CAST X-ray telescope was repurposed to design telescopes for IAXO. An optimal focal length of 5 m for the telescopes was found by taking the estimated telescope figure error and angular size of the sun into account. Producing coated substrates for IAXO requires a different approach than Athena due to the substrates being curved instead of flat. A method and possible design of a coating chamber has been described; the coating non-uniformity due to substrate curvature being addressed with a distance sensor in the coating chamber.

\section{Outlook}
A natural extension of the work described in this thesis would be further investigations into the ability of iridium coatings to mitigate the roughness of the Cr/Ir interface underneath. A rough chromium surface might not be necessary for iridium to mitigate roughness, so it could be used to improve surface roughness on substrates. Both SPO substrates and NuSTAR glass substrates have a surface roughness of $\sigma_{rms}\approx$ 0.45 nm, which traditionally limits the coatings deposited on top to have $\sigma_{rms}\geq$ 0.45 nm. To make new telescopes that reach higher energies than NuSTAR, the interface roughness should significantly decrease in order create multilayers with thinner d-spacings and abiding the rule of thumb $\sigma_{rms} \leq d/6$. A great contender for soft gamma-ray focusing is the WC/SiC material combination that can reach an interface roughness of $\sigma_{rms}\leq$ 0.15 nm using an ultra-flat and highly polished quartz substrate. Such a quartz substrate coated with a WC/SiC multilayer with d-spacings of 1.5 nm have shown Bragg reflections up to 635 keV at the ID15 beamline at ESRF\cite{FernandezPerea:2012fj,FernandezPerea:2013jb}. To apply those coatings to the current flight mirror technology of either SPO substrates or NuSTAR glass substrates requires a method to reduce the surface roughness, and the roughness mitigating qualities of iridium might be the answer.

One major obstacle in obtaining precise multilayer coatings on curved glass substrates is the non-uniformity that comes as a result of the glass curvature. I believe the problem can be overcome by having the cathode power supply be output adjusted as the substrate passes by the cathode, with the adjustment corresponding to the glass curvature. The cathode power supplies at DTU Space can be output adjusted on an analog input port on the back of each power supply by varying the input voltage on that port between 0-10 V.

The X-ray telescopes worked on during this PhD are in completely separate stages of development and deployment. The CAST helioscope has been upgraded with the X-ray telescope and a one-year measurement campaign has begun. It is the hope of the collaboration that a detection will be seen with the new upgrade. Unfortunately the helioscope will likely be decommissioned at the end of 2015, but the telescope has already worked as a proof of concept for the IAXO X-ray telescopes. The Athena mission was selected for a 2028 launch, and with an estimated 10 years required to build it, the European Space Agency is hard at work defining the production facilities required for the optic.
